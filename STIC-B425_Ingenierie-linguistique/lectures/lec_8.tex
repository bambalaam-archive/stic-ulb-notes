\lecture{8}{mercredi 25 mars 2020}
\vspace{-1.2cm}

\section{Aspects sémantiques}

\textbf{Sémantique}: Dérivé du grec ancien \textgreek{σημαντικός} (sêmantikos) "signifié" \\

Postulat : le sens d'une expression linguistique peut être capturé par une structure formelle. Comme pour les aspects morpho-lexicaux et syntaxiques, nous utiliserons des langages de représentation sémantique.\\

\textbf{Compréhension}

La représentation morpho-syntaxique d'un texte est souvent insuffisante pour:

\begin{itemize}
    \item Répondre aux questions d'un examen
    \item Décider quel plat commander au restaurant
    \item Comprendre un manuel d'utilisation
    \item Réaliser qu'on a été insulté
    \item Suivre une recette de cuisine...
\end{itemize}

\subsection{Prérequis}

Exemple: système automatique de recommandation de restaurants pour étudiants Erasmus \\

\textbf{Vérifiabilité}

Exemple: "Le Campouce sert-il des pizzas ?"

Servir(Campouce, Pizzas) $\rightarrow$ Oui / Non / Je ne sais pas \\

Relation entre une phrase et la réalité : valeur de vérité d'une représentation. Vérification à l'aide d'une base de connaissance.\\

\textbf{Non-ambiguïté}

Exemple: "J'aimerais manger au Cimetière" \\

La pertinence de la réponse dépendra de l'interprétation choisie par le système. L'absence d'ambiguïté n'est pas incompatible avec un certain degré d'imprécision/implicite.\\

\textbf{Forme canonique}

Exemple:

"Y a-t-il des pizzas au Campouce ?" / "Le menu du Campouce inclut-il des pizzas ?"/ "Peut-on manger une pizza au Campouce ?"\\

Syntaxe très différente mais sens similaire. Représentation unique (simplification). Word sense disambiguation\\

\textbf{Inférence}

Exemple:

"Les amateurs de pizzas peuvent-ils trouver leur bonheur au Campouce ?"

Capacité à tirer des conclusions valides à propos de propositions non-explicites. \\

"J'aimerais trouver un restaurant où manger une pizza dans les alentours de l'unif."

Servir(x, Pizzas)

Proche(x, ULB)\\

\textbf{Expressivité}

Exemple: "En cas de fringale, où puis-je me sustenter dans les parages de mon Alma Mater?"\\

Un langage de représentation sémantique doit être suffisamment expressif pour modéliser n'importe quel énoncé linguistique. Mais c'est probablement trop utopique pour l'instant?\\

\subsection{Logique du premier ordre}

First-Order Logic (FOL). Langage de représentation remplissant les cinq prérequis énoncés plus haut.\\

Exemple: LeftOf(x, y)

Termes (objets, x et y) et relations (prédicats, LeftOf)\\

\textbf{Termes}

Représentation d'objets du réel:

\begin{itemize}
    \item Constantes: Campouce, Pizzas, Cimetière
    \item Fonctions: Commune(ULB), Origine(Pizzas)
    \item Variables: x,y (objets anonymes ou génériques)\\
\end{itemize}

\textbf{Relations}

Prédicats liant plusieurs objets entre eux:

\begin{itemize}
    \item Servir(Campouce, Pizzas)
    \item Proche(ULB, Cimetière d'Ixelles)\\
\end{itemize}

Aussi utilisé pour encoder une catégorie: Restaurant(Campouce)\\

\textbf{Connecteurs logiques}

"Je n'ai que 5€ \textbf{et} pas beaucoup de temps."

$ \text{Avoir(Locuteur, CinqEuros)} \land \neg \text{Avoir(Locuteur, BeaucoupDeTemps)}$ \\

"J'aimerais manger chinois \textbf{ou} grec."

$ \text{Vouloir(Locuteur,NourritureChinoise)} \lor \text{Vouloir(Locuteur, NourritureGrecque)}$ \\

\textbf{Quantificateurs}

\hspace{2cm} $\exists \text{: il existe}$ \hspace{2cm} $\forall \text{: pour tout}$\\

"Une pizzeria dans le coin du Cimetière"

$\exists x: \text{Restaurant(x)} \land \text{Servir(x, Pizzas)} \land \text{Proche(x, Cimetière\_d'Ixelles)}$\\

"Toutes les pizzerias servent des pizzas."

$ \forall x: \text{Pizzeria(x)} \rightarrow \text{Servir(x, Pizzas)} $ \\

\textbf{Exercices}

« Le Gauguin est un bar » : $$Bar(\text{"Gauguin"})$$

« Le Snack 44 sert des sandwiches » : $$Servir(\text{"Snack 44"}, Sandwiches)$$

« Je veux manger en terrasse, ou alors pas cher » : $$Vouloir(Locuteur, x) \land Restaurant(x) \land ( Avoir(x, Terrasse) \lor \neg Etre(x, Cher) )$$

« J'aime les restos italiens et japonais » : $$Restaurant(x, Italien) \lor Restaurant(x, Japonais)$$

« Peut-on trouver un restaurant végétarien à Ixelles ? » : $$ \exists x: Restaurant(x) \land Type(x, Vegetarien) \land CodePostal(x, 1050) $$

« Les cafés du Cimetière sont proches de l'ULB » : $$ \forall x Cafe(x) : Proche(x, \text{"Cimetiere d'Ixelles"}) \rightarrow Distance(x, \text{"ULB"}) < 5km $$

\newpage

\subsection{Applications}

\textbf{Sémantique lexicale}

Capitaliser les mots comme Restaurant ou Pizzas ne suffit pas pour capturer leur sens.\\

Un peu de vocabulaire:

\begin{itemize}
    \item Lexème : couple (forme + signification)
    \item Lexique : ensemble de lexèmes
    \item Lemme : forme dictionnaire\\
\end{itemize}

Rappel : lemmatisation VS racinisation\\

\textbf{Sens des mots}

\begin{itemize}
\item \textbf{Homonymie:} mots qui ont la même forme orale ou écrite mais des sens différents
\item \textbf{Polysémie:} mot ou expression qui a plusieurs sens ou significations différentes
\item \textbf{Métonymie:} utilise un mot pour signifier une idée distincte mais qui lui est associée
\item \textbf{Homophonie:} identité de sons représentés par des signes différents.
\item \textbf{Homographie:} mots qui s'écrivent de la même manière, tout en se prononçant ou non de façon différente\\
\end{itemize}

\textbf{Relations sémantiques}

\begin{itemize}
\item \textbf{Synonymie:} relation entre deux signifiants et un seul signifié
\item \textbf{Antonymie:} mot qui a un sens opposé à un autre
\item \textbf{Hyponymie / Hyperonymie / Taxonomie:} relation sémantique d'un lexème à un autre selon laquelle l'extension du premier est incluse dans l'extension du second
\item \textbf{Méronymie:} relation sémantique entre mots, lorsqu'un terme désigne une partie d'un second terme\\
\end{itemize}

\textbf{Synset}

Ensembles de quasi-synonymes liés à un sens

Correspond à un concept du réel, distinct de sa réalisation linguistique (cf. arbitraire du signe)

Définition en extension (liste de mots) et en compréhension (glose)

Taxonomie = arbre des hyperonymes\\

\textbf{Word Sense Disambiguation}

Tâche qui consiste à examiner des mots en contexte et à déterminer le sens de chacun

Sous-discipline majeure du TAL au vu des enjeux liés à la désambiguïsation sémantique:

Exemple:
\begin{itemize}
    \item Traduction automatique (avocat → avocado/lawyer)
    \item Recherche d'information en ligne (jaguar vs Jaguar)
    \item Synthèse vocale (signe des temps vs cygne d'étang)\\
\end{itemize}

\textbf{Approche supervisée}

Échantillon lexical présélectionné : mots et sens associés dans un lexique

Nombre limité de possibilités (cibles pour l'apprentissage automatique)

Corpus annoté à la main pour les différents sens

Limite: algorithme spécifique à chaque mot\\

\textbf{Approche non supervisée}

Désambiguïsation de tous les mots d'un texte à partir d'un lexique (e.g. WordNet)

Similaire au POS tagging, mais avec un tagset beaucoup plus étendu (un par lemme)

Plus robuste mais nécessite énormément de données pour capturer les nuances

Limite : il est peu probable que le corpus d'entraînement contienne tous les mots\\
