\lecture{9}{jeudi 30 avril 2020}
\vspace{-1.2cm}

\section{Artificial Intelligence}

Overview d'impact sur le métier de la com. Pas un nouveau terme, date des années 50.
Intelligence artérielle - a amené les machines à effectuer les tâches répétitives tout d'abord. Dans les années
80, on est arrivé à machine learning = machine commence à analyser les choses à partir des données, va en
fonction de son analyse apporter la réponse. On va encore plus loin avec deep learning, avec la reproduction
des nuerons qui permettent aux réseaux des machines d'apprendre les choses sans l'intervention de l'être
humain. Les professionnels sont préoccupés que les machines peuvent dépasser l'être humain, on ne pourra
plus prendre la main sur la société dans laquelle on vit = scénario catastrophique.
Ex : TESLA comme exemple de l'intelligence artificielle.
Les algorithmes sur Internet c'est aussi de machine learning. Ex : sélection des films sur Nexflix.
Deep learning - les machines apprennent les choses par elles-mêmes. Ex : ReCaptha de Google permet de
nourrir les algorithmes et affiner leurs critères. Machine learning = algorithmes, choix parmi la base des
données. Deep mearing = algorithmes qui permettent d'apprendre les choses et prendre les décisions
indépendantes.
Comment algorithme reconnait image ? C'est une simple séquence des choix.
Facebook inversit beaucoup dans l'AI, en particulier sur la reconnaissanca faciale - « ... added a photo you
might be in ». Snapchat aussi a une grosse base de données des nos visages.
GAN = Generated Adversarial Network, des architectures (comme deep learning) constitué de deux réseaux en
compétition, mais qui ont des rôles différents. Avec ça, on est sur de la science-fiction complète où on arrive de
créer les images qui sont tout à fairt réaliste. Deux réseaux : générateur, qui créer le contenu ; discriminateur,
qui vérifie le contenu.
Ex : vidéo de Barack Obama en train de dire des choses qu'il n'a jamais dit

L'industrie pornographique est la plus développée en termes de GAN. Ex : Nathalie Portman dans la porno.
Elon Musk a développé une société qui s'appelle Neuralink qui veut combattre contre l'AI en reliant l'être
humain à des ordinateurs qui vont augmenter sa capacité physique et sa capacité intellectuelle.
Ex : Rembrandt digital exposition
