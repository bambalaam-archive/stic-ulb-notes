\lecture{2}{jeu 20 fev 2020}
\vspace{-1.2cm}

\section{Evolution of ads}

Cannes Lions (Cannes for publicity)
This is our kodak moment - publicity NEEDS a shift to digital

Before:
Passive ads, fixed living room
Impact:
    Develop a concept
    Broadcast a campaign
    Main focus is paid media
    Digital Production
    Concept declination to specialists

Today:
Ads can come from anywhere
Influence:
    Develop an ecosystem
    Starting from storytelling
    (...)

Born from storytellers
Now creating influential digital content

The power of "word of mouth"


Evolution des startegies publicitaires
    Deux evolutions
        Technologie et médias
        Les moeurs (habitudes du consommateur)

    I: Presence (Affiche / Presse)
    II: Positionning (Radio / Logos / Television)
    III: Impact (360)
    IV: Interaction (Internet)
    V: Participation (Social Networks)

    1: Awareness
    2: Superiority
    3: Product difference
    4: Brand difference (Dash VS OMO - Basic Ad VS "show-off")
    5: Adv = Spectacle (Nike, Apple)
    6: Relation (Attract people to your website, not everything is stated on the base advert)
    7: Meaningful (you need a meaningful reason to buy the product: ecology, economy, fair-trade, bio, fair treatment of employees, etc)

    Media Power VS Creativity Power VS Content power VS Data power

Concepts from before:
    You have a slate, a certain ammount of space to show your brand, with creativity.
    Concept = a creative way of saying a message

Today:
    More complex structures, not restricted to an advert space, or a billboard/tv. It's a whole program, with multiple layers. Needs more people than just advertising people, you need developers, etc etc.
    Adapt your ad to the platform you're using.
    Being reactive. You need to go fast to react to the hottest trend or news. "Social Listening".
    Concept = a creative way of linking content to your audiences

Three main evolutions
    Meaningful brands
    Coach Attitude
    Participative Marketing

"We live in a world of content overload. In this world, only brands that form more meaningful connections with people will prosper"
CEO of Havas Group

Before:
    Functional / Cost-Advantage
    Aspirational / Differentiation
Now:
    Meaningful / Make a difference

Meaningful brands generate higher KPIs

3 pillars
    personal benefits (you FEEL better, it's good for you, it improves your life)
    collective benefits (the role of the company in society)
    functional benefits (you NEED it because it works and delivers a service that is incredibly useful)

Pyramid
         ---Values---
    ---Emotional Benefits---
  ------Functional Benefits ------
---Product Attributes (differentiating) ---


Coach Attitude
    Nike: Yesterday you said tomorrow
    "we are not in the business of supporting a media industry, we are in the business of connecting with customers"

    Less focus on the product and more on the customers
    Giving the tools to help people reaching their objectives (health, beauty, career, etc)

    Marque coach doit:
        - se fixer un objectif
            - e.g. acceptation de soi, affirmation de soi, dépassement de soi
        -
        -

    Enabler VS Mentor
        Mentor: booste, dicte, surveille, tire vers le haut, incite à se surpasser
        Enabler: conseille, suggère, aide, soutient, compatit

    Outils
        Longtemps, les outils se limitaient au magazine de marque ou au paquet de céréales.
        Nouvelles modalités relationneles avec le digital
        5 leviers du digital au service du coaching
            - Plus d'accessibilité (smart devices, internet, etc)
            - Plus de récurrence et de rapidité (emails, newsletters, notifications)
            - Plus d'intelligence collective (crowdsourcing, crowdfunding, collectivity)
            - Plus de connaissance et de personnalisation (connected devices, data, give more insight to the customer)
            - Plus de dialogue (chat, social media, etc)
        Coacher par le contenu
        Coacher par le programme (tailor-made)
        Coacher par la donnée (data-personnalisation)
        Coacher par la communauté (people helping eachother)
        Coacher par le jeu (gamification)
        Coacher dans la réalité (IRL coaching)
        Coacher (...)

    Example of coach brand in belgium: KBC helping entrepreneurs to find where there are gaps in the market. Collaboration: people entering zip code and saying what is missing.


Examples:
    - Volvo: green energy for green car
    - Mayo in brasil: provide recipes based on the rest of your cart at the supermarket
    - Nivea: App to track when your kid starts to go super far away like a dummy
    - REI co-op opt outside: go outside instead of going to the shops for black friday
