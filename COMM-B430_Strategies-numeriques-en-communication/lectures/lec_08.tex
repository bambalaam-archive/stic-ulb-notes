\lecture{8}{jeudi 23 avril 2020}
\vspace{-1.2cm}

\section{PMO Influence Mix Model}

Théorie développée par un professeur de Stanford (Itamar Simonson) sur la valeur de l'information et de l'influence. Il a mis en place un modèle qu'il appelle "l'influence mix". Il résume la décision d'un potentiel consommateur en une combinaison de 3 facteurs, PMO:

\begin{itemize}
    \item \textbf{Prior Preferences}: Préférences et expériences personnelles, convictions. Exemple: "je pense que les marques allemandes sont plus solides", "je préfère la marque X à la marque Y", "je n'aime pas ce restaurant".
    \item \textbf{Information about Marketers}: Influence de la publicité, web, stratégies marketing. Exemple: brand promise, brand territory, brand activation.
    \item \textbf{Input from Others}: Avis d'autres consommateurs, connus (amis, influenceurs) ou pas (commentaires dans des sites internet, reviews youtube, etc).\\
\end{itemize}

Il est important de garder en tête que, pour chaque type de produit, ces 3 facteurs ont des pondérations différentes. Il est logique que pour des achats dans l'alimentaire ou dans les besoins de première nécessité le P et le M sont beaucoup plus prépondérants (on ne va pas regarder sur YouTube ce que les autres gens pensent d'une marque de lait ou d'un dentifrice), tandis que pour des achats dans la restauration, les voitures ou l'électronique, l'avis d'autres gens (le O) a une influence beaucoup plus importante.

\section{Native Advertising}

On a vu dans des chapitres précédents que la publicité imprimée et digitale commence à perdre du terrain. Les journaux papier ont une baisse drastique de popularité et la compétition des médias online est très forte. On a aussi le facteur du "banner blindness": les consommateurs ne se focalisent plus sur les pubs, ils ont appris à les ignorer assez efficacement (sans parler des gens qui utilisent des AdBlockers). On est descendu en dessous de 1\% des lecteurs web qui cliquent sur une banner publicitaire, et 94\% des gens cliquent pour ignorer les publicités pre-roll devant les vidéos.

On a également vu dans le chapitre précédent que l'avis de gens externes devient très prisé par les gens avant l'achat d'un produit.\\

En réaction à tous ces éléments est né le Native Advertising: du contenu sponsorisé, pertinent pour l'utilisateur, qui n'interrompt pas son expérience de lecture, et qui s'intègre dans l'environnement éditorial qui l'accueille.\\

On détecte 3 types de Native Advertising:

\begin{itemize}
    \item Branded Content/Publisher Partnerships
    \item In-feed Native Distribution/Native Display
    \item Content Recommendation/Content Discovery
\end{itemize}
